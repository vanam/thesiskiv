%
% Draw black "slugs" whenever a line overflows, so that we can spot it easily.
%
\overfullrule=1cm

% Fix Package inputenc Error: Unicode char ≈ (U+2248) not set up for use with LaTeX.
\DeclareUnicodeCharacter{2248}{$\approx$}

%%% The field of all real and natural numbers
\DeclareMathOperator{\R}{\mathbb{R}}
% \DeclareMathOperator{\N}{\mathbb{N}}
\DeclareMathOperator{\Z}{\mathbb{Z}}

\DeclareMathOperator{\X}{\textbf{X}}            % Data matrix
\DeclareMathOperator{\Sc}{\textbf{S}}           % Scatter matrix
\DeclareMathOperator{\x}{\textbf{x}}            % Data vector
\DeclareMathOperator{\priorm}{\textbf{m}}       % Prior mean
\DeclareMathOperator{\params}{\bm{\theta}}      % Parameters
\DeclareMathOperator{\mean}{\bm{\mu}}           % Mean
\DeclareMathOperator{\emean}{\bar{\x}}          % Empirical mean
\DeclareMathOperator{\cov}{\mathbf{\Sigma}}     % Covariance matrix (Note: Sigma not bold when using \bm...)
\DeclareMathOperator{\icov}{\mathbf{\Lambda}}   % Inverse Covariance matrix
\DeclareMathOperator{\covt}{\text{cov}\,}       % Covariance matrix (text)
\DeclareMathOperator{\ca}{\textbf{c}}           % Customer assignments
\DeclareMathOperator{\z}{\textbf{z}}            % Table assignments
\DeclareMathOperator{\stud}{\mathcal{T}\,}      % Student's t-distribution
\DeclareMathOperator{\ind}{\mathbb{I}}          % Indicator function
\DeclareMathOperator{\bet}{B\,}                 % Beta distribution
\DeclareMathOperator{\dir}{\text{Dir}\,}        % Dirichlet distribution
\DeclareMathOperator{\cat}{\text{Cat}\,}        % Categorical distribution
\DeclareMathOperator{\mult}{\text{Mu}\,}        % Multinomial distribution
\DeclareMathOperator{\norm}{\mathcal{N}\,}      % Normal distribution
\DeclareMathOperator{\wi}{\text{Wi}\,}          % Wishart distribution
\DeclareMathOperator{\iw}{\text{IW}\,}          % Inverse Wishart distribution
\DeclareMathOperator{\niw}{\text{NIW}\,}        % Normal Inverse Wishart
\DeclareMathOperator{\aic}{\text{AIC}\,}        % Akaike information criterion
\DeclareMathOperator{\bic}{\text{BIC}\,}        % Bayesian information criterion
\DeclareMathOperator{\clusters}{\mathbf{\Omega}}% Set of clusters
\DeclareMathOperator{\classes}{\textbf{C}}      % Set of classes
\DeclareMathOperator{\purity}{\text{purity}\,}  % Purity measure
\DeclareMathOperator{\rand}{\text{RI}\,}        % Rand index
\DeclareMathOperator{\mi}{\text{I}\,}           % Mutual information
\DeclareMathOperator{\entropy}{\text{H}\,}      % Entropy
\DeclareMathOperator{\nmi}{\text{NMI}\,}        % Normalized mutual information
\DeclareMathOperator{\homogeneity}{h}           % Homogeneity
\DeclareMathOperator{\completeness}{c}          % Completeness
\DeclareMathOperator{\vmeasure}{V}              % V-measure
\DeclareMathOperator{\nvmeasure}{NV}            % NV-measure
\DeclareMathOperator{\precision}{P\,}           % Precision
\DeclareMathOperator{\recall}{R\,}              % Recall
\DeclareMathOperator{\fmeasure}{F}              % F-measure
\DeclareMathOperator{\tp}{T\!P}                 % True positive
\DeclareMathOperator{\tn}{T\!N}                 % True negative
\DeclareMathOperator{\fp}{F\!P}                 % False positive
\DeclareMathOperator{\fn}{F\!N}                 % False negative

%%% Useful operators for statistics and probability
\DeclareMathOperator{\pr}{p}
\DeclareMathOperator{\E}{\mathbb{E}\,}
% \DeclareMathOperator{\var}{\textrm{var}}
% \DeclareMathOperator{\sd}{\textrm{sd}}
\DeclareMathOperator{\tr}{\textrm{tr}}   % Trace of a matrix

%%% Transposition of a vector/matrix
\newcommand{\T}[1]{#1^\top}

%%% Various math goodies
\newcommand{\goto}{\rightarrow}
\newcommand{\gotop}{\stackrel{P}{\longrightarrow}}
\newcommand{\maon}[1]{o(n^{#1})}
\newcommand{\abs}[1]{\left|{#1}\right|}
\newcommand{\dett}[1]{\left|{#1}\right|}
\newcommand{\dint}{\int_0^\tau\!\!\int_0^\tau}
\newcommand{\isqr}[1]{\frac{1}{\sqrt{#1}}}

%%% Various table goodies
\newcommand{\pulrad}[1]{\raisebox{1.5ex}[0pt]{#1}}
\newcommand{\mc}[1]{\multicolumn{1}{c}{#1}}

%%% argmin/argmax      https://tex.stackexchange.com/questions/5223/command-for-argmin-or-argmax
\DeclareMathOperator*{\argmin}{arg\,min}
\DeclareMathOperator*{\argmax}{arg\,max}

% Small matrix with () surrounding
\newenvironment{psmallmatrix}
  {\left(\begin{smallmatrix}}
  {\end{smallmatrix}\right)}

\newcommand{\cpp}{C\texttt{++} }
